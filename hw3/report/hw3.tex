\documentclass[12pt, a4paper]{article}


% A pretty common set of packages
\usepackage[margin=2.5cm]{geometry}
\usepackage[T1]{fontenc}
\usepackage{graphicx}
\usepackage{amssymb}
\usepackage{amsmath}
\usepackage{bm}
\usepackage{color}
\usepackage{float}
\usepackage{bm}
\usepackage{physics}
\usepackage{subcaption}

\DeclareRobustCommand{\uvec}[1]{{%
  \ifcsname uvec#1\endcsname
     \csname uvec#1\endcsname
   \else
    \bm{\hat{\mathbf{#1}}}%
   \fi
}}
\newcommand{\olsi}[1]{\,\overline{\!{#1}}} % overline short italic

\usepackage[colorlinks=true, 
    linkcolor=blue,          % color of internal links
    citecolor=blue,        % color of links to bibliography
    filecolor=blue,      % color of file links
    urlcolor=blue]{hyperref}

\title{[16-833] Homework 3 : Written Report}
\author{Bharath Somayajula}
\date{\today}

\begin{document}

\maketitle

\tableofcontents
\section{Measurement function}
\subsection{Odometry}
\subsubsection{Measurement Function}
Since the odometry results in a change in position, the measurement function is simply
\[h_o(\mathbf{r}^t, \mathbf{r}^{t+1}) = \mathbf{r}^{t+1} - \mathbf{r}^t\]
\subsubsection{Jacobian}
The Jacobian of measurement function is
\[H_o(\mathbf{r}^t, \mathbf{r}^{t+1}) = \begin{bmatrix}
  -1 & 0 & 1 & 0\\
  0 & -1 & 0 & 1\\
\end{bmatrix}\]
\subsection{Landmark}
Since landmarks are measured using the relative position of the robot and landmark, the measurement function is simply
\[h_l(\mathbf{r}^t, \mathbf{l}^{k}) = \mathbf{l}^{k} - \mathbf{r}^t\]
\subsubsection{Jacobian}
The Jacobian of measurement function is
\[H_l(\mathbf{r}^t, \mathbf{l}^{k}) = \begin{bmatrix}
  -1 & 0 & 1 & 0\\
  0 & -1 & 0 & 1\\
\end{bmatrix}\]
\end{document}