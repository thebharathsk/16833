\documentclass[12pt, a4paper]{article}


% A pretty common set of packages
\usepackage[margin=2.5cm]{geometry}
\usepackage[T1]{fontenc}
\usepackage{graphicx}
\usepackage{amssymb}
\usepackage{amsmath}
\usepackage{bm}
\usepackage{color}
\usepackage{float}
\usepackage{bm}
\usepackage{physics}
\usepackage{subcaption}

\DeclareRobustCommand{\uvec}[1]{{%
  \ifcsname uvec#1\endcsname
     \csname uvec#1\endcsname
   \else
    \bm{\hat{\mathbf{#1}}}%
   \fi
}}
\newcommand{\olsi}[1]{\,\overline{\!{#1}}} % overline short italic

\usepackage[colorlinks=true, 
    linkcolor=blue,          % color of internal links
    citecolor=blue,        % color of links to bibliography
    filecolor=blue,      % color of file links
    urlcolor=blue]{hyperref}

\title{[16-833] Homework 4 : Written Report}
\author{Bharath Somayajula}
\date{\today}

\begin{document}

\maketitle

\tableofcontents

\section{Iterative Closest Point}
\subsection{Projective Data Association}
For a source point to have a valid correspondence in target vertex map, it's projected coordinates must fall within the bounds of the vertex map and have a positive depth. Therefore, assuming that the u, v coordinates are rounded to integers and the height and width are zero-indexed, we get:
\[0 \leq  u < W\]
\[0 \leq  v < H\]
\[d > 0\]

\end{document}
